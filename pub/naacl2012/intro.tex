
\Section{intro}{Introduction}
% -- Temporal Comparison
%(intro to systems)
Temporal resolution is a well-studied area, with a number of successful systems
	implemented.

%(system X) ...
\todo{Prior work}

%(ambiguities)
\todo{2 types of ambiguity}

%(segue into approach)
Diverging from the common rule-based approach to temporal resolution,
	we explore employing distantly supervised probabilistic parsing
	to learn to parse these expressions.

% -- Approach
We assume that temporal phrases -- such as \tp{Last Friday} -- have
	associated with them not only a grounded time (the date of the previous
	Friday), but also a latent parse.
This would assign a meaning to the terms \tp{last} and \tp{Friday}; the
	meaning of the compound phrase would then be a compositional combination
	of the two components.

In this sense, the task can be framed as a distantly supervised parsing
	problem.
Given an input phrase and the constraint that the parse must ground to the
	given gold time, the task becomes to infer a latent parse during
	training, and to infer this parse and the grounding it refers to during
	evaluation.

Such a compositional grammar is developed, and an EM-esque algorithm is
	employed to infer latent parses and learn the relevant parameters for
	the grammar.

% -- Theoretical backing
\todo{theoretical backing}

% -- Semantic Parsing Comparison
\todo{compare to Percy/Luke}

\Section{related}{Related Work}
Our approach takes inspiration from \me, both in the bootstrapping training
  methodology and the temporal grammar.
Our foremost contributions over this prior work are:
  (i) the utilization of a discriminative parser trained with rich features;
  (ii) simplifications to the temporal grammar which nonetheless maintain
       high accuracy;
  and
  (iii) experimental results on 6 different languages with state-of-the-art
        performance on both datasets on which prior work exists.

% -- Semantic Parsing Comparison
As in this previous work,
  our approach draws inspiration from work on semantic parsing.
The latent parse parallels the formal semantics in previous work,
	e.g., Montague semantics.
Supervised approaches to semantic parsing prominently include
	\newcite{key:1996zelle-semantics},
	\newcite{key:2005zettlemoyer-semantics},
	\newcite{key:2005kate-semantics}, 
	\newcite{key:2007zettlemoyer-semantics}, 
	\textit{inter alia}.
For example, \newcite{key:2007zettlemoyer-semantics} learn a mapping from
	textual queries to a logical form.
Importantly, the logical form of these parses contain all of the predicates
  and entities used in the parse -- unlike the annotation provided in our case,
  where a grounded time can correspond to any of a number of parses.
%(distantly supervised)
Along this line, recent work by \newcite{key:2010clarke-semantics} and 
	\newcite{key:2011liang-semantics} relax supervision 
	to require only annotated answers rather than full logical forms.

% -- Rule-Based Temporal Comparison
%(rule based)
Related work on interpreting temporal expressions has focused on constructing
	hand-crafted interpretation rules
	% Random, TERESO, Marta's Friend, Edinburgh
	\cite{key:2000mani-temporal,key:2003saquete-temporal,key:2004puscasu-temporal,key:2010grover-temporal}.  
Of these, \sys{HeidelTime} \cite{key:2010strotgen-temporal} and
	\sys{SUTime} \cite{key:2012chang-temporal} provide a particularly strong
	comparison in English.

%(probabilistic)
Recent probabilistic approaches to temporal resolution include
	\newcite{key:2010uzzaman-temporal},    %TRIPS/TRIOS - probabilistic
	who employ a parser to produce deep logical forms, in conjunction with
	a CRF classifier.
In a similar vein,
	\newcite{key:2010kolomiyets-temporal} %KUL - partly probabilistic
	employ a maximum entropy classifier to detect the location and temporal
	type of expressions; the grounding is then done via deterministic rules.
	
%(spanish)
In addition, there has been work on parsing Spanish expressions;
  UC3M \cite{2010vicente-uc3m} produce the strongest results on the
  \tempeval\ corpus.
%(multilingual)
Of the systems entered in the original task,
  TIPSem \cite{key:2010llorens-tipsem} was the only system to perform bilingual
  interpretation for English and Spanish.
Both of the above systems rely primarily on hand-built rules.




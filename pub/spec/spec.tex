\documentclass{article}
\usepackage{amssymb,amsmath,amsthm} 
\usepackage[margin=1.25in]{geometry}
\input{std-macros}
\begin{document}

\title{Time}
\author{}
\date{\today}
\maketitle

\def\second#1{\ensuremath{#1_{second}}}
\def\minute#1{\ensuremath{#1_{minute}}}
\def\hour#1{\ensuremath{#1_{hour}}}
\def\day#1{\ensuremath{#1_{day}}}
\def\week#1{\ensuremath{#1_{week}}}
\def\month#1{\ensuremath{#1_{month}}}
\def\year#1{\ensuremath{#1_{year}}}

%%%%%%%%%%%%%%%%%%%%%%%%%%%%%%%%%%%%%%%%%%%%%%%%%%%%%%%%%%%%%%%%%%%%%%%%%%%%%%%%
% PRIMITIVES
%%%%%%%%%%%%%%%%%%%%%%%%%%%%%%%%%%%%%%%%%%%%%%%%%%%%%%%%%%%%%%%%%%%%%%%%%%%%%%%%
\Section{primitives}{Primitives}
\Subsection{timeformat}{Time Format}

A time expression is modeled as an ordered sequence of ranges:
\begin{equation*}
	\left< (x_{0_0}, x_{0_1}), (x_{1_0}, x_{1_1})\dots \right>
\end{equation*}
where $x_i$ denotes the start of the range (inclusive) and $y_i$ denotes
	the end of the range (exclusive).

The variables in this sequence can either be grounded to a pair of dates
	(effectively, {\em a miliseconds since 1970} sort of representation);
however, the representation can also allow for free variables.
Of course, a hybrid between bounded (grounded) and unbounded (free, abstract)
	variables is also possible.
This is represented as below:
\begin{equation*}
	\lambda_{x,y,\dots}
		. \left< (x, y), \dots \right>
\end{equation*}

Two broad classes of legal operations can be done on such a sequence:
\begin{enumerate}
	\item {\em Piecewise operations} take two sequences of the same length
		and performs the operation on the elements of each sequence piecewise.
	\item {\em One-to-many operations} take two sequences, one of which is
		a singleton, and applies the operation to each element of the non-singleton
		using the singleton as the second argument for each.
\end{enumerate}
Note that {\em one-to-one} operations are a special case of the second.
Also note that operating on sequences which are not of the same size is
	undefined in the general case.

Notation-wise, the norm ($|r|$) of a sequence is defined as the difference
	between the second and the first element, in a {\em miliseconds since
	1970} sort of way.
If the sequence is not a singleton, the norm of a sequence is only defined
	if each element has the same norm.

\Subsection{primitivetimes}{Primitive Times}
A number of terms would be mapped directly to a range.
As a first pass, I think most of these could be hard-coded;
as a second pass, I think the number of such expressions is small enough
	that the alignment between the word and the expression could be learned
	(i.e. to make it portable for different languages).

These, along with examples, are enumerated below:

\begin{center}
\begin{tabular}{l l l}
	{\bf General Form} & {\bf Example} & {\bf Example's Parse} \\
	\hline

	\multicolumn{3}{l}{{\bf dates (grounded)}} \\
		$\left< (x, y) \right>$
		& {\em March 6 2011} & $\left< (03/06/2011, 03/07/2011) \right>$ \\
		& {\em March 2011} & $\left< (03/01/2011, 04/01/2011) \right>$ \\
		& {\em 2011} & $\left< (01/01/2011, 01/01/2012) \right>$ \\
	
	\multicolumn{3}{l}{{\bf repeated dates (grounded)}} \\
		$\left< (x, x+n) : x \in \sX \right>$
		& {\em Friday} & $\left< (x,x+\week{1}) : x \in \{ \T{Fridays} \} \right>$\\
		& {\em March 5th} & $\left< (x,x+\day{1}) : x \in \{ 03/05/* \} \right>$ \\
		& {\em March} & $\left< (x,x+\month{1}) : x \in \{ 03/1/* \} \right>$ \\
	
	\multicolumn{3}{l}{{\bf dates (abstract)}} \\
		$\left< (t, t+n) \right>$
		& {\em today} & $\lambda_t.\left< (t,t+\day{1}) \right>$\\
	
	\multicolumn{3}{l}{{\bf ranges (abstract)}} \\
		$\left< (x, x+n) \right>$
		& {\em Week} & $\lambda_x.\left< (x,x+\week{1}) \right>$\\
		& {\em Hours} & $\lambda_x.\left< (x,x+\hour{1}) \right>$\\
		& {\em Year} & $\lambda_x.\left< (x,x+\year{1}) \right>$\\
\end{tabular}
\end{center}


\Subsection{primitiveoperations}{Primitive Operations}
Operations on primitive times fall into the three general categories
	of shifting the range, expanding the range in some direction,
	and merging ranges (for different meanings of 'merge').
These would be functions mapped to words (or, better, groups of words)
	and would combine with each other in a CCG-esque way.

These operations, in more detail, are enumerated below:

\begin{center}
\begin{tabular}{l l l}
	{\bf General Form} & {\bf Example} & {\bf Example's Parse} \\
	\hline

	\multicolumn{3}{l}{{\bf shifts}} \\
	$\lambda_{x,r}.\left< (x_0-|r|, x_0) \right>$
		& {\em Day before March 7th 2011} 
			& $\left< (03/06/2011,03/07/2011) \right>$\\
		& {\em Day before today} 
			& $\lambda_{t}.\left< (t-\day{1},t) \right>$\\
		& {\em Last week} 
			& $\lambda_t.\left< (t-\week{1},t) \right>$ \\
		& {\em yesterday} 
			& $\lambda_t.\left< (t-\day{1},t) \right>$ \\
	$\lambda_{x,r}.\left< (x_1, x_1+|r|) \right>$
		& {\em Day after March 7th 2011} 
			& $\left< (03/08/2011,03/09/2011) \right>$\\
		& {\em Day after today} 
			& $\lambda_{x}.\left< (t+\day{1},t+\day{1}+\day{1}) \right>$\\
	
%	\multicolumn{3}{l}{{\bf floor/ceil}} \\
	$\lambda_{x,r}.\left< (x_0, x_0+|r|) \right>$
		& {\em First day of March} 
			& $\left< (x,x+\day{1}) : x \in \{ 03/1/* \} \right>$ \\
		& {\em First month of the year} 
			& $\left< (x,x+\month{1}) : x \in \{ 01/1/* \} \right>$ \\
	$\lambda_{x,r}.\left< (x_1-|r|, x_1) \right>$
		& {\em Last day of March} 
			& $\left< (x+\month{1}-\day{1},x+\month{1}) 
				: x \in \{ 03/1/* \} \right>$ \\
		& {\em Last week of March} 
			& $\left< (x+\month{1}-\week{1},x+\month{1}) 
				: x \in \{ 03/1/* \} \right>$ \\
	
	\multicolumn{3}{l}{{\bf intersect}} \\
	\multicolumn{3}{l}{
			$\lambda_{x,y}.\left< (x_0, x_0+min(|x|,|y|)) : x \in X \cap Y \right>$}\\
		& {\em April of 2007} 
			& $\left< (04/01/2007,05/01/2007) \right>$ \\
		& {\em April last year} 
			& $\lambda_t.\left< x 
			  : x \in \{ (04/01/*,05/01/*) \cap (t-\year{1},t) \} \right>$ \\
		& {\em Friday last week} 
			& $\lambda_t.\left< x 
			  : x \in \{ \T{Fridays} \cap (t-\week{1},t) \} \right>$ \\
	
	\multicolumn{3}{l}{{\bf cons}} \\
	$\lambda_{x,y}.\left< (x_0, y_1) \right>$
		& {\em 5/3/2010 to 5/10/2010} 
			& $\lambda_{t}.\left< (05/03/2010,05/10/2010) \right>$ \\
		& {\em since yesterday} 
			& $\lambda_{t,u}.\left< (t+\day{1},u) \right>$ \\
		& {\em until tomorrow} 
			& $\lambda_{t,u}.\left< (u,t+\day{1}) \right>$ \\
		& {\em before tomorrow} 
			& $\lambda_{t,u}.\left< (u,t+\day{1}) \right>$ \\
		& {\em yesterday until tomorrow} 
			& $\lambda_{t}.\left< (t-\day{1},t+\day{1}) \right>$ \\
\end{tabular}
\end{center}

%%%%%%%%%%%%%%%%%%%%%%%%%%%%%%%%%%%%%%%%%%%%%%%%%%%%%%%%%%%%%%%%%%%%%%%%%%%%%%%%
% RULES
%%%%%%%%%%%%%%%%%%%%%%%%%%%%%%%%%%%%%%%%%%%%%%%%%%%%%%%%%%%%%%%%%%%%%%%%%%%%%%%%
\Section{rules}{Combination Rules}

\end{document}

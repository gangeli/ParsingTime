\documentclass{article}
\usepackage{amssymb,amsmath,amsthm} 
\usepackage[margin=1.0in]{geometry}
\input{std-macros}
\begin{document}

\title{Time}
\author{}
\date{\today}
\maketitle

\def\second#1{\ensuremath{#1_{second}}}
\def\minute#1{\ensuremath{#1_{minute}}}
\def\hour#1{\ensuremath{#1_{hour}}}
\def\day#1{\ensuremath{#1_{day}}}
\def\week#1{\ensuremath{#1_{week}}}
\def\month#1{\ensuremath{#1_{month}}}
\def\quarter#1{\ensuremath{#1_{quarter}}}
\def\year#1{\ensuremath{#1_{year}}}
\def\sp{\hspace{5mm}}

%%%%%%%%%%%%%%%%%%%%%%%%%%%%%%%%%%%%%%%%%%%%%%%%%%%%%%%%%%%%%%%%%%%%%%%%%%%%%%%%
% PRIMITIVES
%%%%%%%%%%%%%%%%%%%%%%%%%%%%%%%%%%%%%%%%%%%%%%%%%%%%%%%%%%%%%%%%%%%%%%%%%%%%%%%%
\Section{primitives}{Primitives}
\Subsection{timeformat}{Time Format}

A time expression is modeled as an ordered sequence of ranges:
\begin{equation*}
	\left< (x_{0_0}, x_{0_1}), (x_{1_0}, x_{1_1})\dots \right>
\end{equation*}
where $x_i$ denotes the start of the range (inclusive) and $y_i$ denotes
	the end of the range (exclusive).

The variables in this sequence can either be grounded to a pair of dates
	(effectively, {\em a miliseconds since 1970} sort of representation);
however, the representation can also allow for free variables.
Of course, a hybrid between bounded (grounded) and unbounded (free, abstract)
	variables is also possible.
This is represented as below:
\begin{equation*}
	\lambda_{x,y,\dots}
		. \left< (x, y), \dots \right>
\end{equation*}

Two broad classes of legal operations can be done on such a sequence:
\begin{enumerate}
	\item {\em Piecewise operations} take two sequences of the same length
		and performs the operation on the elements of each sequence piecewise.
	\item {\em One-to-many operations} take two sequences, one of which is
		a singleton, and applies the operation to each element of the non-singleton
		using the singleton as the second argument for each.
\end{enumerate}
Note that {\em one-to-one} operations are a special case of the second.
Also note that operating on sequences which are not of the same size is
	undefined in the general case.

Notation-wise, the norm ($|r|$) of a sequence is defined as the difference
	between the second and the first element, in a {\em miliseconds since
	1970} sort of way.
If the sequence is not a singleton, the norm of a sequence is only defined
	if each element has the same norm.

\Subsection{primitivetimes}{Primitive Times}
A number of terms would be mapped directly to a range.
As a first pass, I think most of these could be hard-coded;
as a second pass, I think the number of such expressions is small enough
	that the alignment between the word and the expression could be learned
	(i.e. to make it portable for different languages).

These, along with examples, are enumerated in \reftab{primitives}.

\begin{table}
\begin{center}
\begin{tabular}{l l l}
	{\bf General Form} & {\bf Example} & {\bf Example's Parse} \\
	\hline
	\hline

	\multicolumn{3}{l}{{\bf dates (grounded)}} \\
		$\left< (x, y) \right>$
		& {\em March 6 2011} & $\left< (03/06/2011, 03/07/2011) \right>$ \\
		& {\em March 2011} & $\left< (03/01/2011, 04/01/2011) \right>$ \\
		& {\em 2011} & $\left< (01/01/2011, 01/01/2012) \right>$ \\
	
	\multicolumn{3}{l}{{\bf repeated dates (grounded)}} \\
		$\left< (x, x+n) : x \in \sX \right>$
		& {\em Friday} & $\left< (x,x+\week{1}) : x \in \{ \T{Fridays} \} \right>$\\
		& {\em March 5th} & $\left< (x,x+\day{1}) : x \in \{ 03/05/* \} \right>$ \\
		& {\em March} & $\left< (x,x+\month{1}) : x \in \{ 03/1/* \} \right>$ \\
	
	\multicolumn{3}{l}{{\bf dates (abstract)}} \\
		$\left< (t, t+n) \right>$
		& {\em today} & $\lambda_t.\left< (t,t+\day{1}) \right>$\\
	
	\multicolumn{3}{l}{{\bf ranges (abstract)}} \\
		$\left< (x, x+n) \right>$
		& {\em Week} & $\lambda_x.\left< (x,x+\week{1}) \right>$\\
		& {\em Hours} & $\lambda_x.\left< (x,x+\hour{1}) \right>$\\
		& {\em Year} & $\lambda_x.\left< (x,x+\year{1}) \right>$\\
	
	\multicolumn{3}{l}{{\bf numbers}} \\
		& {\em 24} & $24$ \\
		& {\em seven} & $7$ \\
	\hline
\end{tabular}
\end{center}
\caption{
\label{tab:primitives}
Grounded and ungrounded primitive time expressions
}
\end{table}


\Subsection{primitiveoperations}{Primitive Operations}
Operations on primitive times fall into the three general categories
	of shifting the range, expanding the range in some direction,
	and merging ranges (for different meanings of 'merge').
These would be functions mapped to words (or, better, groups of words)
	and would combine with each other in a CCG-esque way.

These operations, in more detail, are enumerated in table \reftab{ops}.

\begin{table}
\begin{center}
\begin{tabular}{l l l}
	{\bf General Form} & {\bf Example} & {\bf Example's Parse} \\
	\hline
	\hline

	\multicolumn{3}{l}{{\bf shifts}} \\
	$\lambda_{x,r}.\left< (x_0-|r|, x_0) \right>$
		& {\em Day before March 7th 2011} 
			& $\left< (03/06/2011,03/07/2011) \right>$\\
		& {\em Day before today} 
			& $\lambda_{t}.\left< (t-\day{1},t) \right>$\\
		& {\em Last week} 
			& $\lambda_t.\left< (t-\week{1},t) \right>$ \\
		& {\em yesterday} 
			& $\lambda_t.\left< (t-\day{1},t) \right>$ \\
	$\lambda_{x,r}.\left< (x_1, x_1+|r|) \right>$
		& {\em Day after March 7th 2011} 
			& $\left< (03/08/2011,03/09/2011) \right>$\\
		& {\em Day after today} 
			& $\lambda_{x}.\left< (t+\day{1},t+\day{1}+\day{1}) \right>$\\
	
%	\multicolumn{3}{l}{{\bf floor/ceil}} \\
	$\lambda_{x,r}.\left< (x_0, x_0+|r|) \right>$
		& {\em First day of March} 
			& $\left< (x,x+\day{1}) : x \in \{ 03/1/* \} \right>$ \\
		& {\em First month of the year} 
			& $\left< (x,x+\month{1}) : x \in \{ 01/1/* \} \right>$ \\
	$\lambda_{x,r}.\left< (x_1-|r|, x_1) \right>$
		& {\em Last day of March} 
			& $\left< (x+\month{1}-\day{1},x+\month{1}) 
				: x \in \{ 03/1/* \} \right>$ \\
		& {\em Last week of March} 
			& $\left< (x+\month{1}-\week{1},x+\month{1}) 
				: x \in \{ 03/1/* \} \right>$ \\
	
	\multicolumn{3}{l}{{\bf intersect}} \\
	\multicolumn{3}{l}{
			$\lambda_{x,y}.\left< (z_0, z_0+min(|x|,|y|)) : z \in X \cap Y \right>$}\\
		& {\em April of 2007} 
			& $\left< (04/01/2007,05/01/2007) \right>$ \\
		& {\em April last year} 
			& $\lambda_t.\left< x 
			  : x \in \{ (04/01/*,05/01/*) \cap (t-\year{1},t) \} \right>$ \\
		& {\em Friday last week} 
			& $\lambda_t.\left< x 
			  : x \in \{ \T{Fridays} \cap (t-\week{1},t) \} \right>$ \\
	
	\multicolumn{3}{l}{{\bf cons}} \\
	$\lambda_{x,y}.\left< (x_0, y_1) \right>$
		& {\em 5/3/2010 to 5/10/2010} 
			& $\lambda_{t}.\left< (05/03/2010,05/10/2010) \right>$ \\
		& {\em since May 3 2010} 
			& $\lambda_{u}.\left< (05/03/2010,u) \right>$ \\
		& {\em since yesterday} 
			& $\lambda_{t,u}.\left< (t+\day{1},u) \right>$ \\
		& {\em until tomorrow} 
			& $\lambda_{t,u}.\left< (u,t+\day{1}) \right>$ \\
		& {\em before tomorrow} 
			& $\lambda_{t,u}.\left< (u,t+\day{1}) \right>$ \\
		& {\em yesterday until tomorrow} 
			& $\lambda_{t}.\left< (t-\day{1},t+\day{1}) \right>$ \\
	
	\multicolumn{3}{l}{{\bf numeric literals}} \\
	$\lambda_{r,n}.\left< (r_0, r_0+|r|*n) \right>$
		& {\em 24 hours} 
			& $\lambda_x.\left< (x,x+\hour{24}) \right>$\\
		& {\em eight days} 
			& $\lambda_x.\left< (x,x+\day{8}) \right>$\\
	$\lambda_{x,n}.\left< (x_0+|x|*n, x_1+|x|*n) \right>$
		& {\em third day} 
			& $\lambda_t.\left< (t+\day{3},t+\day{1}+\day{3}) \right>$\\
		\hline
\end{tabular}
\end{center}
\caption{
\label{tab:ops}
Operations on time primitives
}
\end{table}

%%%%%%%%%%%%%%%%%%%%%%%%%%%%%%%%%%%%%%%%%%%%%%%%%%%%%%%%%%%%%%%%%%%%%%%%%%%%%%%%
% RULES
%%%%%%%%%%%%%%%%%%%%%%%%%%%%%%%%%%%%%%%%%%%%%%%%%%%%%%%%%%%%%%%%%%%%%%%%%%%%%%%%
\Section{rules}{Combination Rules}
\Subsection{rulesgeneral}{Description}
Each operation is a binary relation, and each takes as arguments either
	a grouned or ungrounded time expression, and returns either a grounded
	or ungrounded time expression.
In this fashion, they should be able to be combined indefinitely.

The exception to this rule is in the case of {\bf cons} when using unbounded
	ranges (e.g. {\em until tomorrow}).
In this case, a free variable appears, which will eventually be mapped to
	either positive or negative infinity.
In practice, I think such expressions could be treated like normal, as it
	seems strange to combine things with 'since' anyways.

The ability to use arbitrary ordering when applying the rules I haven't
	worked through entirely yet.
However; most of the rules seem to either be order independent
	(e.g. intersect), or else take qualitatively different types of arguments
	(e.g. a grounded/ungrounded date, and a range).

\Subsection{rulesexamples}{Examples}
Parses of some real-world time expressions are enumerated in
	table \reftab{examples}.

\begin{table}
\begin{center}
\begin{tabular}{l l l}
	{\bf Expression} & {\bf Time} & {\bf Component} \\
	\hline
	\hline
	{\em third-quarter}
		& $\left< (x+\quarter{3},x+\quarter{4}) : x \in \{ Q*/* \} \right>$ 
			& {\em third-quarter} \\
		& \sp$3$ 
			& \sp{\em third} \\
		& \sp$\left< (x,x+\quarter{1}) : x \in \{ Q*/* \} \right>$ 
			& \sp{\em quarter} \\
	\hline
	{\em this year}
		& $\left< \lambda_t(t,t+\year{1}) \right>$ 
			& {\em this year} \\
		& \sp- 
			& \sp{\em this} \\
		& \sp$\lambda_t\left< (t,t+\year{1})\right>$ 
			& \sp{\em year} \\
	\hline
	{\em a year ago}
		& $\left< \lambda_x(x_0-\year{1},x_0) \right>$ 
			& {\em a year ago} \\
		& \sp- 
			& \sp{\em a} \\
		& \sp$\lambda_t\left< (t,t+\year{1})\right>$ 
			& \sp{\em year} \\
		& \sp$\lambda_{x,r}\left< (x_0-|r|,x_0)\right>$ 
			& \sp{\em ago} \\
	\hline
	{\em \#recently}
		& $\left< \lambda_{x,r}(x_0-|r|,x_0) \right>$ 
			& {\em recently} \\
	\hline
	{\em the year ending June 30, 1990}
		& $\left< (06/31/1989,06/31/1990) \right>$ 
			& {\em the year ending June 30, 1990} \\
		& \sp $\lambda_x.\left< (x, x+\year{1}) \right>$
			& \sp{\em the year} \\
		& \sp\sp-
			& \sp\sp{\em the} \\
		& \sp\sp$\lambda_x.\left< (x, x+\year{1}) \right>$
			& \sp\sp{\em year} \\
		& \sp$\lambda_{x,y}.\left< (x_0,y_1) \right>$ 
			& \sp{\em ending} \\
		& \sp$\left< (06/30/1990,06/31/1990)\right>$ 
			& \sp{\em June 30, 1990} \\
	\hline
	{\em the nine months}
		& $\lambda_x.\left<(x,x+\month{9}) \right>$ 
			& {\em the nine months} \\
		& \sp-
			& \sp{\em the} \\
		& \sp9
			& \sp{\em nine} \\
		& \sp$\lambda_x.\left<(x,x+\month{1}) \right>$ 
			& \sp{\em months} \\
	\hline
	{\em the latest quarter}
		& $\left< (x-\quarter{1},x) : x \in \{ Q*/* \} \right>$ 
			& {\em the latest quarter} \\
		& \sp-
			& \sp{\em the} \\
		& \sp$\lambda_{x,r}.\left< (x_0-|r|,x_0) \right>$
			& \sp{\em latest} \\
		& \sp$\left< (x,x+\quarter{1}) : x \in \{ Q*/* \} \right>$ 
			& \sp{\em quarter} \\
	\hline
	{\em\#the first nine months}
		& $\lambda_x.\left< (x,x+\month{9}) \right>$
			& {\em the first nine months} \\
		& \sp- 
			& \sp{\em the} \\
		& \sp $\lambda_{x,r}.\left< (x_0, x_0+|r|) \right>$
			& \sp{\em first} \\
		& \sp $\lambda_x.\left< (x,x+\month{9}) \right>$
			& \sp{\em nine months} \\
		& \sp\sp 9
			& \sp\sp{\em nine} \\
		& \sp\sp $\lambda_x.\left< (x,x+\month{1}) \right> $
			& \sp\sp{\em months} \\
	\hline
\end{tabular}
\end{center}
\caption{
\label{tab:examples}
Example parses
}
\end{table}

%%%%%%%%%%%%%%%%%%%%%%%%%%%%%%%%%%%%%%%%%%%%%%%%%%%%%%%%%%%%%%%%%%%%%%%%%%%%%%%%
% LEARNING
%%%%%%%%%%%%%%%%%%%%%%%%%%%%%%%%%%%%%%%%%%%%%%%%%%%%%%%%%%%%%%%%%%%%%%%%%%%%%%%%
\Section{learning}{Learning}
\Subsection{data}{Training Data}
The format described above should be compatible with TimeML.
An sequence of intervals $\left< (a_i,b_i) \right>$ could be expressed
	in the timeML format using the {\tt beginPoint}, {\tt endPoint},
	and {\tt freq} attributes of the {\tt TIMEX3} tag.

Training data would be obtained either from the TimeBank corpus, or from
	GUTime output on some large unlabelled corpus trained in a semi-supervised
	way.
In fact, both could probably be used.


\Subsection{training}{Training}
A resonably first approximation would be simple parsing, though.
The upside is that it's probably efficient; 
	the downside is that every expression is bound to one and only one word,
	making distinctions like {\em last week} versus {\em the last week} hard
	to capture.

\end{document}
